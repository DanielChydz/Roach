\begin{itemize}
\item sformułowanie problemu
\item osadzenie tematu w kontekście aktualnego stanu wiedzy (\english{state of the art}) o poruszanym problemie
\item  studia literaturowe \cite{bib:artykul,bib:ksiazka,bib:konferencja} -  opis znanych rozwiązań (także opisanych naukowo, jeżeli problem jest poruszany w publikacjach naukowych), algorytmów, 
\end{itemize}

Wzory  
\begin{align}
y = \frac{\partial x}{\partial t}
\end{align}
jak i pojedyncze symbole $x$ i $y$  składa się w trybie matematycznym.


%\begin{Definition}\label{def:1}
%Definicja to zdanie (lub układ zdań) odpowiadające na pytanie o strukturze „co to jest a?”. Definicja normalna jest zdaniem złożonym z 2 członów: definiowanego (łac. definiendum) i definiującego (łac. definiens), połączonych spójnikiem definicyjnym („jest to”, „to tyle, co” itp.). 
%\end{Definition}
%
%\begin{Theorem}[Pitagorasa]\label{t:pitagoras}
%W dowolnym trójkącie prostokątnym suma kwadratów długości przyprostokątnych jest równa kwadratowi długości przeciwprostokątnej tego trójkąta. 
%\end{Theorem}
%
%\begin{Example}[generalizacja]\label{ex:generalizacja}
%Przykładem generalizacji jest para: zwierzę i pies. Pies jest zwierzęciem. Pies jest uszczegółowieniem pojęcia zwierzę. Zwierzę jest uogólnieniem pojęcia pies.
%\end{Example}