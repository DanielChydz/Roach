\begin{itemize}
\item[PID] regulator Proporcjonalno-Całkująco-Różniczkujący (ang.~\english{Proportional–Integral–Derivative})
\item[MPC] regulator oparty na modelu predykcyjnym (ang.~\english{Model Predictive Control})
\item[DC] prąd stały (ang.~\english{Direct Current})
\item[LED] dioda emitująca światło (ang.~\english{Light Emitting Diode})
\item[3D] 3 wymiary (ang.~\english{3 dimensions})
\item[Wi-Fi] sieć bezprzewodowa (ang.~\english{Wireless Fidelity})
\item[RTOS] system czasu rzeczywistego (ang.~\english{Real Time Operating System})
\item[FDM] modelowanie przez nakładanie stopionego materiału (ang.~\english{Fused Deposition Modeling})
\item[DLP] modelowanie przez utwardzanie materiału światłem (ang.~\english{Digital Light Processing})
\item[PLA] filament do druku 3D, polilaktyd (ang.~\english{Polylactic Acid})
\item[LIDAR] technologia mapowania przestrzeni 3D bazująca na detekcji światła i pomiarze odległości (ang.~\english{LIDAR})
\item[$u$] sygnał sterujący obiektem wykonawczym w układzie sterowania/regulacji
\item[$u(i)$] sygnał sterujący w $i$-tej chwili czasu (pętli)
\item[$u\textsubscript{max}$] maksymalna wartość sygnału sterującego
\item[$EI$] suma błędu (ang. \english{Error Integral})
\item[$p\textsubscript{max}$] maksymalna liczba pulsów silnika na pętlę na biegu jałowym, wartość stała
\item[$\alpha$] współczynnik maksymalnej prędkości silników, w przedziale $\left<1; \frac{p\textsubscript{max}}{100}\right>$ 
\end{itemize}