\label{ch:results}
Nie wszystkie założenia początkowe zostały spełnione. Główne zmiany dotyczą fizycznego wykonania pojazdu. Zrezygnowano z~osi przedniej i~4~kół na rzecz koła podporowego. Usunięto także czujnik laserowy i~głośnik sygnalizacyjny. Rama pojazdu, wbrew założeniu, nie przypomina do końca pojazdu osobowego, co spowodowane jest niskimi umiejętnościami modelowania 3D autora. Pojazd zgodnie z wymogami jest modułowy, ale ostatecznie modułowość ta nie została wykorzystana, i~pomiary przeprowadzono tylko dla jednego rozmiaru koła o średnicy 30~mm, mimo że nawet kod ESP32 wspiera zmianę średnicy koła.

Realizacja projektu, mimo zmian konstrukcyjnych, pokazuje istotne aspekty praktycznego zastosowania teorii sterowania w~rzeczywistych warunkach. Ostateczna forma pojazdu, chociaż odbiega od pierwotnych założeń, podkreśla elastyczność procesu projektowego w~adaptacji do ograniczeń technicznych i~umiejętności wykonawczych. Skupienie się na modułowości, nawet jeśli nie zostało pełni wykorzystane, stanowi solidną bazę dla przyszłych modyfikacji i~ulepszeń. Takie podejście pozwala na łatwe dostosowanie projektu do nowych wymagań lub eksperymentów bez konieczności projektowania urządzenia od podstaw.

Poza wymienionymi, wszystkie założenia projektowe zostały spełnione, a~sam projekt został pomyślnie ukończony z~pozytywnymi wynikami. Dla wybranych parametrów PID (Tabela \ref{tab:finalneparametrypid}), pojazd z~wysoką dokładnością w~granicach 1.5~cm dojeżdża do zadanej odległości, zachowując jednocześnie synchronizację pozycji silników, a~tym samym przejechanego dystansu kół. Średni uchyb między kołami w~stanie ustalonym wynosi 0.35~cm, co w praktyce jest prawie niezauważalne podczas przejazdu.

Jako że dla wszystkich regulatorów PID wzmocnienie różniczkujące jest zerowe, można mówić raczej o~sterowaniu przy pomocy regulatorów PI, ponieważ mimo że technicznie część różniczkująca została zaimplementowana, to nie zaistniała potrzeba jej użycia podczas doboru nastaw.

Wadą są oscylacje prędkości zadanej silnika śledzącego ze względu na wysokie wzmocnienie członu całkującego równe K\textsubscript{I}=3, jednak nie przekłada się to na oscylacje rzeczywistego układu. Wraz ze zmniejszeniem tego wzmocnienia, co prawda zanikają oscylacje, jednak zwiększa się uchyb między pozycją koła prowadzącego a~śledzącego.

Sam model prowadzący-śledzący posiada zasadniczą wadę niewidoczną w~warunkach doświadczalnych, mianowicie w~przypadku gdy silnik śledzący z~jakiegoś powodu przestanie nadążać, silnik prowadzący nie będzie dążył do wyrównania prędkości i~pozycji, tym samym drastycznie zwiększając uchyb. Jedną z metod korekcji~byłoby dodanie sygnału niezależnego do koła prowadzącego, tym samym dodając czwarty regulator PI.

Podsumowując, projekt ten dostarcza cennych wniosków o~praktycznym zastosowaniu teorii sterowania w~realnych aplikacjach inżynierskich. Eksperymentalna natura projektu, pomimo pewnych ograniczeń, demonstruje możliwości adaptacji i~innowacji w dziedzinie robotyki i~automatyzacji, oraz umiejętności młodych automatyków-robotyków.