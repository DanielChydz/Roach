\label{ch:wstep}

Robotyka to obszar badawczy i~techniczny poświęcony teorii, konstrukcji oraz praktycznym zastosowaniom robotów. Elementami wykonawczymi układów zrobotyzowanych są najczęściej silniki lub siłowniki, te drugie nierzadko napędzane wewnętrznie silnikami. Rodzajem maszyny zamieniającej jeden rodzaj energii --- w~robotyce najczęściej elektryczną --- na energię mechaniczną, czego celem jest wprawienie w~ruch elementów ruchomych.

W przypadku prostych układów[def. \ref{def:uklad}], takich jak podajnik taśmowy napędzany pojedynczym silnikiem, dokładność[\ref{def:dokladnosc}] sterowania nie ma wysokiego priorytetu. Najważniejsze jest, by element znajdujący się na taśmie przejechał z~punktu A~do punktu B~z~pewną prędkością, a~jego położeniem zajmą się inne czujniki. Jednak gdy silnik napędza ramię robota, pojazd lub drona, ważne jest, by utrzymywał stałą prędkość i/lub wykonywał określoną ilość obrotów.

\begin{Definition}[Układ sterowania]\label{def:uklad}
   Układ służący do kontroli pożądanego urządzenia przy pomocy wybranego zasobu narzędzi, w~tym zamkniętej pętli z~ujemnym sprzężeniem zwrotnym[def. \ref{def:petla}].
\end{Definition}

\begin{Definition}[Pętla]\label{def:petla}
    Typ struktury układu umożliwiający mu reakcję na informację zwrotną o~jego stanie (sprzężenie zwrotne) pochodzącą z czujników.
\end{Definition}

\begin{Definition}[Dokładność]\label{def:dokladnosc}
    Stopień zgodności pomiędzy rzeczywistymi wartościami a~wartościami określonymi lub mierzonymi.
\end{Definition}

Z tego powodu, jednym z~wyzwań, z~jakimi mierzyli się pionierzy automatycy-robotycy jest dokładne sterowanie tworzonymi przez siebie układami. Jest to kwestia o~tyle istotna, że gdy odpowiedź układu odbiega --- nawet w niewielkim stopniu --- od wartości zadanej, staje się on znacząco trudniejszy w~użytkowaniu (sterowaniu), a w skrajnych przypadkach bezużyteczny. W~przypadku ramienia robota, brak dokładności sterowania doprowadzi układ do osiągnięcia innej lokalizacji niż pożądana. W~przypadku drona, prawdopodobnie całkowicie uniemożliwi lot z~powodu nierówności sił nośnych. Zaś w~przypadku pojazdów autonomicznych, przejechanie odległości innej niż zadana przez operatora. Jako rozwiązanie tego problemu, powstał poddział robotyki zwany odometrią[def. \ref{def:odometria}]. 

\begin{Definition}[Odometria]\label{def:odometria}
    Dział nauk technicznych na pograniczu robotyki i~miernictwa, zajmujący się użyciem różnego rodzaju czujników w~celu oszacowania położenia ruchomego układu względem pozycji startowej w przestrzeni fizycznej.
\end{Definition}

Współcześni automatycy-robotycy będący na początku swojej ścieżki edukacji/kariery, lub zajmujący się robotyką jedynie hobbystycznie, również mierzą się prędzej czy później z~problemem dokładności sterowania układu. W~obecnych czasach istnieje wiele sposobów rozwiązania go.

Celem niniejszej pracy inżynierskiej jest zaprojektowanie i~wykonanie fizycznego modelu platformy mobilnej (zwanej dalej pojazdem) oraz aplikacji do jej kontroli. Następnie zastosowanie i~przetestowanie eksperymentalne wybranej metody zwiększenia dokładności układu z~użyciem ujemnego sprzężenia zwrotnego. Na końcu otrzymane wyniki zostaną przeanalizowane pod kątem użyteczności i skuteczności wybranej metody w warunkach rzeczywistych.

Praca podzielona została na następujące rozdziały\cite{bib:wymaganiapracy}:
\begin{enumerate}
    \item Wstęp --- wprowadzenie do tematu, krótkie omówienie istoty problemu, zakres pracy, opis rozdziałów.
    \item Analiza tematu --- omówienie tematu w kontekście aktualnego stanu wiedzy o poruszanym problemie (ang. \english{state of art}), oraz szczegółowe jego sformułowanie.
    \item Założenia projektowe --- opis wymagań postawionych przy tworzeniu projektu wraz z uzasadnieniem wyboru, a także lista użytych narzędzi.
    \item Projekt i wykonanie --- szczegółowe omówienie sposobu wykonania modelu pojazdu, układu elektronicznego, aplikacji, kodu mikroprocesora i narzędzi pobocznych. Każdemu elementowi odpowiada podrozdział.
    \item Weryfikacja i walidacja --- zastosowane metody badawcze i wykonane eksperymenty, oraz napotkane i usunięte błędy.
    \item Podsumowanie i wnioski --- skomentowanie uzyskanych wyników pod kątem osiągnięcia założonych celów, analiza i dalsze kroki.
\end{enumerate}