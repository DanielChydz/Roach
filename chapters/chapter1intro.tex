\label{ch:wstep}

Robotyka to obszar badawczy i techniczny poświęcony teorii, konstrukcji oraz praktycznym zastosowaniom robotów. Elementami wykonawczymi układów zrobotyzowanych są najczęściej silniki lub siłowniki, te drugie nierzadko napędzane wewnętrznie silnikami. Silnik jest rodzajem maszyny zamieniającym jeden rodzaj energii --- w robotyce najczęściej elektryczną --- na energię mechaniczną, czego celem jest wprawienie w ruch elementów ruchomych.

W przypadku prostych układów, takich jak podajnik taśmowy napędzany pojedynczym silnikiem, precyzja sterowania nie ma wysokiego priorytetu. Najważniejsze jest, by element znajdujący się na taśmie przejechał z punktu A do punktu B z pewną prędkością, a jego położeniem zajmą się inne czujniki. Jednak gdy silnik napędza ramię robota, pojazd lub drona, ważne jest, by utrzymywał stałą prędkość i/lub wykonywał określoną ilość obrotów.

Z tego powodu, jednym z wyzwań, z jakimi mierzyli się pionierzy automatycy-robotycy jest precyzyjne sterowanie tworzonymi przez siebie układami. Jest to kwestia o tyle istotna, że gdy odpowiedź układu odbiega --- nawet w niewielkim stopniu --- od wartości zadanej, staje się on znacząco trudniejszy w użytkowaniu (sterowaniu), a w skrajnych przypadkach bezużyteczny.

Jako rozwiązanie tego problemu, powstał poddział robotyki zwany odometrią. Jest to dział na pograniczu robotyki i miernictwa, zajmujący się użyciem różnego rodzaju czujników w celu oszacowania położenia ruchomego układu względem pozycji startowej w przestrzeni fizycznej.

Współcześni automatycy-robotycy będący na początku swojej ścieżki edukacji/kariery, lub zajmujący się nią jedynie hobbystycznie, również mierzą się prędzej czy później z problemem precyzji sterowania układu.