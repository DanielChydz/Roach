% Tutaj proszę umieszczać swoje pakiety, makra, ustawienia itd.


 
%%%%%%%%%%%%%%%%%%%%%%%%%%%%%%%%%%%%%%%%%%%%%%%%%%%%%%%%%%%%%%%%%%%%%
% listingi i fragmentu kodu źródłowego 
% pakiet: listings lub minted
% % % % % % % % % % % % % % % % % % % % % % % % % % % % % % % % % % % 

% biblioteka listings
\usepackage{listings}
\usepackage{xcolor}
\lstset{
    language=C++,                 % język programowania
    backgroundcolor=\color{white}, % kolor tła
    commentstyle=\color{green},    % styl komentarzy
    keywordstyle=\color{blue},     % styl słów kluczowych
    numberstyle=\small\color{gray}, % styl numeracji linii
    stringstyle=\color{purple},    % styl stringów
    basicstyle=\ttfamily\small,    % podstawowy styl czcionki
    breakatwhitespace=false,       % łamanie linii (false = łamie w dowolnym miejscu)
    breaklines=true,               % automatyczne łamanie linii
    captionpos=b,                  % pozycja tytułu (t=top, b=bottom)
    keepspaces=true,               % zachowuje spacje w kodzie (potrzebne do zachowania wcięć)
    numbers=left,                  % gdzie umieścić numerację linii
    numbersep=5pt,                 % jak daleko numeracja linii jest od kodu
    showspaces=false,              % pokazuje spacje za pomocą podkreśleń; przesłania 'showstringspaces'
    showstringspaces=false,        % nie pokazuje spacji w stringach
    showtabs=false,                % pokazuje tabulacje za pomocą podkreśleń
    tabsize=2,                     % ustawia domyślny rozmiar tabulatora
    frame=single,                  % dodaje ramkę do kodu
    rulecolor=\color{black},       % kolor ramki
    title=\lstname,                % pokazuje nazwę pliku pod kodem
    captionpos=t,                  % tytuł nad kodem
    escapeinside={\%*}{*)},        % jeśli chcesz dodać LaTeXa do swojego kodu
    morekeywords={*,...},          % jeśli chcesz dodać więcej słów kluczowych do zestawu
}
\renewcommand\lstlistingname{Źródło}
\renewcommand\lstlistlistingname{Źródła}

\usepackage{enumerate}
\usepackage{enumitem}
\setitemize{noitemsep,topsep=0pt,parsep=0pt,partopsep=0pt}
\setenumerate{noitemsep,topsep=0pt,parsep=0pt,partopsep=0pt}
\usepackage{floatrow}
\floatsetup[table]{capposition=top}

% % % % % % % % % % % % % % % % % % % % % % % % % % % % % % % % % % % 
% pakiet minted
%\usepackage{minted}

% pakiet wymaga specjalnego kompilowania:
% pdflatex -shell-escape main.tex
% xelatex  -shell-escape main.tex

%\usepackage[chapter]{minted} % [section]
%%\usemintedstyle{bw}   % czarno-białe kody 
%
%\setminted % https://ctan.org/pkg/minted
%{
%%fontsize=\normalsize,%\footnotesize,
%%captionpos=b,%
%tabsize=3,%
%frame=lines,%
%framesep=2mm,
%numbers=left,%
%numbersep=5pt,%
%breaklines=true,%
%escapeinside=@@,%
%}

%%%%%%%%%%%%%%%%%%%%%%%%%%%%%%%%%%%%%%%%%%%%%%%%%%%%%%%%%%%%%%%%%%%%%